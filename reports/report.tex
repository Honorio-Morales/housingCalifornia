\documentclass{article}
\usepackage[utf8]{inputenc}
\usepackage{graphicx}
\usepackage{booktabs}
\usepackage{geometry}
\geometry{margin=1in}
\title{Taller 1.1 -- Housing California: Regresión Lineal y Polinomial}
\author{Equipo}
\date{\today}

\begin{document}
\maketitle

\section*{Resumen}
Este informe resume el Análisis Exploratorio de Datos (EDA) y la comparación entre un modelo de Regresión Lineal Múltiple y Modelos de Regresión Polinomial (grado 2 y 3) usados para predecir el valor medio de la vivienda (`MedHouseVal`) en el dataset California Housing.

\section{Dataset}
El dataset local utilizado está en `data/california_housing.csv`. Se emplearon las features: `MedInc`, `HouseAge`, `AveRooms`, `AveBedrms`, `Population`, `Latitude`, `Longitude`.

\section{Análisis Exploratorio (EDA)}
\begin{figure}[ht]
  \centering
  \includegraphics[width=0.8\textwidth]{figures/heatmap_correlation.png}
  \caption{Heatmap de correlaciones entre features y la variable objetivo.}
\end{figure}

\begin{figure}[ht]
  \centering
  \includegraphics[width=0.9\textwidth]{figures/scatter_top_features.png}
  \caption{Scatter plots: variables más correlacionadas frente a `MedHouseVal`.}
\end{figure}

\section{Modelado y Métricas}
Se entrenaron los siguientes modelos y se evaluó su desempeño en el conjunto de prueba (mismas particiones que en `src/train.py`):

\begin{table}[ht]
\centering
\begin{tabular}{lcc}
\toprule
Modelo & $R^2$ & RMSE \\
\midrule
Regresión Lineal Múltiple & 0.5751 & 0.7462 \\
Regresión Polinomial (grado 2) & 0.6070 & 0.7176 \\
Regresión Polinomial (grado 3) & 0.5393 & 0.7770 \\
\bottomrule
\end{tabular}
\caption{Comparación de métricas (R$^2$ y RMSE) en el conjunto de prueba.}
\end{table}

\section{Real vs Predicho y Análisis de Residuos}
\begin{figure}[ht]
  \centering
  \includegraphics[width=0.9\textwidth]{figures/pred_resid_linear.png}
  \caption{Real vs Predicho y residuos -- Regresión Lineal.}
\end{figure}

\begin{figure}[ht]
  \centering
  \includegraphics[width=0.9\textwidth]{figures/pred_resid_poly2.png}
  \caption{Real vs Predicho y residuos -- Polinomial grado 2.}
\end{figure}

\begin{figure}[ht]
  \centering
  \includegraphics[width=0.9\textwidth]{figures/pred_resid_poly3.png}
  \caption{Real vs Predicho y residuos -- Polinomial grado 3.}
\end{figure}

\section{Conclusiones preliminares}
El modelo polinomial de grado 2 presenta la mejor $R^2$ y menor RMSE en el conjunto de prueba (posible mejor ajuste sin overfitting evidente respecto a grado 3). Se recomienda:\\
- Revisar feature engineering y regularización si se busca mejorar generalización.\\
- Analizar residuos por segmento geográfico (Latitude/Longitude) para detectar patrones espaciales.

\section*{Compilación}
Para compilar este documento localmente (desde la raíz del proyecto):
\begin{verbatim}
cd reports
pdflatex report.tex
\end{verbatim}

\end{document}
